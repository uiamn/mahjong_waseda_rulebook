\documentclass[uplatex]{jsarticle}

\begin{document}

\begin{itemize}
    \item 条件付き東南戦で東4局終了時に持ち点が25500点以上の方がゐない場合は南入します.南入時は誰かが25,500点を超えた時点で終了します.
    \item 喰ひ断あり後付けありのありありルールです.
    \item 25,000点持ち30,000点返し,順位ウマは20-60です.同点時は起家により近い方を上位とします.
    \item 誰かの持ち点が55,000点以上になった時点で終了です.
    \item 1本場1,500点です.
    \item 聴牌連荘で形式聴牌があります.ノーテン罰符は場に3,000点です.
    \item 親番での和了時もしくは流局時には次局の親番を放棄することができます.下家の方が次局の親番となり,本場は 0 本場にリセットされます.
    \item 東場は東と西,南場は南と北が常時役牌になります.ピンフの雀頭として使ふことはできません.
    \item 特殊牌として以下の牌が各1枚ずつ入つてゐます(計5枚).
        \begin{itemize}
            \item トパーズポッチ(白): 立直後にツモつた場合に強制和了となります(これをオールマイティツモと呼びます).河に切つても戻すことができます.和了牌が複数種類ある場合,裏ドラ,カン裏ドラを見てから好きな待ち牌に取ることができます.また祝儀1枚が付きます.通常の白として扱つた場合にも祝儀は発生します.
            \item ルビー(5ピン),アクアマリン(5ソウ),エメラルド(5マン):ドラで,それぞれ祝儀が1,2,3枚付きます.鳴いても祝儀が付きます.
            \item シルバ(5ピン):2ハン分のドラが付きますが,祝儀はありません.
        \end{itemize}
    \item 上記祝儀牌の他,一発,裏ドラ,カン裏ドラ,トビにも祝儀1枚が付きます.
    \item 特殊役として次の2つがあります.
        \begin{itemize}
            \item ジュエル:上記祝儀牌を3枚以上使用した場合ジュエルといふ\textbf{1ハンの和了役}が付きます.
            \item フルジュエル:上記祝儀牌を全て使用した場合本役満になります.
        \end{itemize}
        ただし,上記何れの場合においても,オールマイティツモ時のトパーズポッチを祝儀牌にはカウントしません.
    \item 役満祝儀はツモ4枚オール,ロン8枚です.上記祝儀と複合します.
    \item 祝儀は1枚10,000点相当です.
    \item 13ハン以上で数へ役満ですが,役満祝儀はありません.
    \item 30符4ハン,60符3ハンは満貫です.
    \item 連風牌の雀頭の加符点は2符です.
    \item カンドラは先めくりです.
    \item 0点未満でトビです(0点丁度は続行).持ち点900点以下では立直はできません.
    \item 箱下精算はありません.
    \item 四風連打,九種九牌,四カン,四家立直は途中流局で,親は流れ,本場が増えます.
    \item ダブロンがあります.本場,供託,親権は上家取りです.トビの祝儀は両者に支払ひます.
    \item オープン立直(2,000点),フリテン立直,立直後の見逃しありです.ツモ番のない立直はできません.オープン立直への放銃は,手詰まりは本役満扱ひ,さもなくば戻して和了放棄となります.
    \item チョンボは満貫払ひで,親の場合はオーラス時以外親権が流れます.
    \item 副露は晒す前かつ打牌前であれば1,000点供託で取り消しが可能です.言ひ間違ひ,晒し間違ひは副露が可能であればノーペナルティです(可能でなければ1,000点供託).晒した後での副露の取り消しはできず,何を切つても喰ひ替へになる場合は和了放棄です.
    \item 倒牌がない誤ツモ,誤ロンは和了放棄です.
    \item 副露はポン優先です.
    \item オーラスの親は着順に関はらず和了やめ,聴牌やめが可能です.
    \item 流し満貫があります.ただし自分が副露や立直をしてゐる場合は成立しません.満貫のツモ和了扱ひです.
    \item 大三元,大四喜,大明カンのパオがあります.ツモ和了時は直撃扱ひの責任払ひ,他家の放銃時は祝儀を含めた折半です.パオ対象者のチョンボのペナルティは,役満確定者への放銃扱ひです.
    \item 役満の複合があります.四暗刻単騎,純正九蓮宝燈,国士無双13面待ちはシングル役満です.ただしいづれも祝儀は役満祝儀と同じです.
    \item 人和があります(動きのない状態で自分の最初の打牌前に他家から和了牌が切られた場合に成立).倍満もしくは通常のロン和了での得点のうち高い方になります.
    \end{itemize}

\end{document}
