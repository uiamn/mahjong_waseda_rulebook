\documentclass[uplatex]{jsarticle}

\begin{document}

\begin{itemize}
    \item 東風戦,喰ひ断あり後付けありのありありルールです.
    \item 25,000点持ち30,000点返し,順位ウマは20-50です.同点時は起家により近い方を上位とします.
    \item 1本場1,500点です.
    \item 聴牌連荘で形式聴牌があります.ノーテン罰符は場に3,000点です.
    \item 東と西が常時役牌になります.ピンフの雀頭として使ふことはできません.
    \item 5マン,5ピン,5ソウに赤と金が1枚ずつ入つてゐます.ドラ扱ひで,赤牌には祝儀1枚,金牌には祝儀2枚が付きます.鳴いても祝儀が付きます.
    \item 白に白ポッチが1枚入つてゐます.立直後にツモつた場合に強制和了となります.河に切つても戻すことができます.当たり牌が複数ある場合,裏ドラ,カン裏ドラを見てから好きな待ち牌に取ることができます.また祝儀1枚が付きます(通常の白として扱つた場合には付きません).
    \item 赤牌,金牌,白ポッチの他,一発,裏ドラ,カン裏ドラにも祝儀が付きます.
    \item 役満祝儀はツモ5枚オール,ロン10枚です.上記祝儀と複合します.
    \item 祝儀は1枚5,000点相当です.
    \item 13ハン以上で数へ役満ですが,役満祝儀はありません.
    \item 30符4ハン,60符3ハンは満貫です.
    \item 連風牌の雀頭の可符点は4符です.
    \item カンドラは先めくりです.
    \item 0点未満でトビです(0点丁度は続行).持ち点900点以下では立直はできません.
    \item トビ賞はありませんが,箱下精算は無限大です.
    \item 四風連打,九種九牌,四カン,四家立直は途中流局で,ノーゲーム扱ひで親は流れず,本場も増えません.
    \item ダブロン,トリロンがあります.本場,供託,親権は上家取りです.
    \item フリテン立直,立直後の見逃しありです.
    \item チョンボはチップ1枚オールでノーゲーム扱ひです.
    \item 副露は晒す前かつ打牌前であれば1,000点供託で取り消しが可能です.言ひ間違ひ,晒し間違ひは副露が可能であればノーペナルティです(可能でなければ1,000点供託).晒した後での副露の取り消しはできず,何を切つても喰ひ替へになる場合は和了放棄です.
    \item 倒牌がない誤ツモ,誤ロンは和了放棄です.
    \item 副露は発声優先です.
    \item オーラスの親は着順に関はらず和了やめ,聴牌やめが可能です.
    \item 流し満貫があります.自分が副露や立直をしてゐても成立し,満貫のツモ和了扱ひで,本場分の加点,供託の回収,親権の移動があります.流局時の手牌に含まれる祝儀牌分と,立直時は裏ドラ・カン裏ドラ分の祝儀が付きます.
    \item 大三元,大四喜,四槓子にパオがあります.ツモ和了時は直撃扱ひの責任払ひ,他家の放銃時は祝儀を含めた折半です.パオ対象者のチョンボのペナルティは,役満確定者への放銃扱ひです.
    \item 役満の複合があります.また,四暗刻単騎,純正九蓮宝燈,国士無双13面待ちはダブル役満です.ただしいづれも祝儀は役満祝儀と同じです.
    \item 十三不塔があります(動きのない1巡目で雀頭があり,他にターツがない場合に成立).本役満扱ひで祝儀が発生します.
    \item 人和があります(動きのない状態で自分の最初の打牌前に他家から和了牌が切られた場合に成立).8ハン役です.
    \end{itemize}

\end{document}
